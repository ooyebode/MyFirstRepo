\documentclass[11pt]{article}

\setlength{\oddsidemargin}{0in}
\setlength{\evensidemargin}{0in}
\setlength{\topmargin}{-0.5in}
\setlength{\headheight}{0pt}
\setlength{\topskip}{0pt}
\setlength{\textwidth}{6.5in}
\setlength{\textheight}{9in}
%\setlength{\parindent}{0pt}
\setlength{\parskip}{1mm}

\usepackage{epsf}

% Magic incantation to allow proper PostScript to PDF conversion:
  \usepackage[T1]{fontenc} % type 1 fonts
  \usepackage{times}       % Times-Roman

\sloppy

\newcommand{\Fwname}{Johar}

\begin{document}

\begin{center} \bf \Large
Johar ``Star'' Interface Interpreter \\
Requirements Specification of the Star GUI
\end{center}

\section{The Main Panel}

See Figure \ref{mainPanelFig}.  The Main Panel is displayed when the
application is started, and remains visible throughout the lifetime of the
application.
\begin{enumerate}
\item At the top is the {\it Menu Bar}.  This shows one menu
  for each command group in the application, and one menu named
  ``Star'' (for services provided by the Star interpreter for all
  Johar applications).  We refer to the non-Star menus as ``application
  menus''.
\item Each application menu is labelled in the Menu Bar by the name of the
  command group.
\item The menu items in each application menu are the commands that are in
  that command group.
\item An application menu item is clickable (not greyed out) if the corresponding
  command is {\it active}.
\item A command is {\it active} at a given point if either it has no
  {\tt ActiveIfMethod}, or it has an {\tt ActiveIfMethod} and that
  method returns {\tt true}.
\item On the left is the {\it Text Display Area}.  This shows all
  text messages sent by the application engine using {\tt showText}
  that are not displayed through some other means.
\item On the right is the {\it Table Area}.  This is a tabbed pane.
\item In the Table Area, there is one tab corresponding to every
  revealed Table currently in the application.
\item The application decides which Tables to conceal and reveal.
\item If, during the processing of a command, the application engine sets
  the top table, then immediately after the command has finished
  processing, the topmost table in the Table Area is the last table to
  have been set as the top table.
\item If, during the processing of a command, the application engine
  does not set the top table, then immediately
  after the command has finished processing, the topmost table in the
  Table Area remains unchanged from the time that the command was issued.
\item Notwithstanding the table set as the topmost table by the app
  engine, the user can (between issuing one command and issuing the
  next command) click on a table tab in order to move that table to
  the top.
\item At the bottom is the {\it Status Bar}.  This shows low-prominence
  messages sent by the application engine using {\tt showText}.
\item On the Star menu, there is one menu item, ``Help''.
\item When the ``Help'' menu item is selected, the Help Box is
  displayed (see below).
\end{enumerate}

\begin{figure}[t]

\centerline{\epsffile{StarGui_mainPanel.eps}}

\caption{
  The main panel.
}
\label{mainPanelFig}
\end{figure}


\section{The Command Dialog Box}

\begin{figure}

\centerline{\epsfxsize=5.5in \epsffile{StarGui_commandDialogBox.eps}}

\caption{
  The command dialog box.
}
\label{commandDialogBoxFig}
\end{figure}

See Figure \ref{commandDialogBoxFig}.
\begin{enumerate}
\item At the top is the {\tt Label} attribute of the command.
\item Under that is one section for every {\it queryable parameter}
  (see below) in the current stage.
\item Many commands have only one stage.  However, if there is
  more than one stage, only the queryable parameters in the current stage
  will appear in the parameter section.
\item There is one section of the box for each queryable parameter in the
  current stage.
\item At the bottom are two to four buttons: {\tt Cancel}, {\tt Previous},
  {\tt Next}, and {\tt OK}.  {\tt Cancel} and {\tt OK} always appear;
  {\tt Previous} and {\tt Next} do not always appear.
\item The {\it Cancel button} always appears and is always enabled.
\item The Cancel button is on the left-hand side of the dialog box.
\item The {\it Previous button}:
  \begin{enumerate}
  \item Appears if there is more than one {\it queryable stage} (see below).
  \item Is enabled if the current stage is not the first queryable stage.
  \end{enumerate}
\item The {\it Next button}:
  \begin{enumerate}
  \item Appears if there is more than one queryable stage.
  \item Is enabled if the current stage is not the last queryable stage.
  \end{enumerate}
\item The {\it OK button}:
  \begin{enumerate}
  \item Always appears.
  \item Is enabled if there are no {\it incomplete stages} (see below).
  \end{enumerate}
\item The OK button is on the right-hand side of the dialog box.
\item A parameter is {\it inactive} if both (a) it has a {\tt ParentParameter},
  and (b) the current value of the {\tt ParentParameter} is not the
  {\tt ParentValue} for this parameter.
\item A {\it queryable parameter} is any parameter
  that is not a {\tt tableEntry} parameter with
  a browsable {\tt SourceTable}.
\item A {\it queryable stage} is a stage that contains some
  queryable parameter.
\item An {\it incomplete stage} is a stage that contains some parameter
  such that: (a) the parameter is not inactive, (b) there is no
  {\tt DefaultValue} or {\tt DefaultValueMethod} for the parameter,
  and (c) the current number of repetitions for which the user has
  selected a value is less than the
  {\tt MinNumberOfReps} for the parameter.

  (That is, any repetitions that the user has not filled in and that
  do not have a default value should be interpreted as repetitions
  that the user does not want to give to the application engine.
  If the number of the remaining repetitions is less than the
  {\tt MinNumberOfReps}, then the user still needs to fill in more
  values.)
\end{enumerate}

\section{Parameter Section of the Command Dialog Box}

\begin{figure}

\centerline{\epsfxsize=5.5in \epsffile{StarGui_parameterSection.eps}}

\caption{
  The parameter section of the command dialog box.
}
\label{parameterSectionFig}
\end{figure}

Every queryable parameter in the current stage is represented by a section
of the command dialog box.  See Figure \ref{parameterSectionFig}.
\begin{enumerate}
\item On the left is the {\tt Label} attribute of the parameter,
  at the top of the left-hand side, followed by a colon.
\item On the right is one subsection for each repetition of the parameter,
  and sometimes a small {\it Add Another} button.
\item Initially (before any user interaction), the number of repetition
  sections will be equal to maximum(1, $m$), where $m$ is the value of
 {\tt MinNumberOfReps} for the parameter.
\item The {\tt Add Another} button:
  \begin{enumerate}
  \item Has the ToolTip ``Add another''.
  \item Is a small button with a plus sign (``{\tt +}'') in it.
  \item Appears only if {\tt MinNumberOfReps} is not equal to
    {\tt MaxNumberOfReps} for this parameter.
  \item Is enabled only if the current number of repetitions of the
    parameter is less than {\tt MaxNumberOfReps} for this parameter.
  \item Is on the left-hand side of the right-hand half of the parameter
    section.
  \end{enumerate}
\item The entire section corresponding to the parameter is greyed out
  if the parameter is inactive.
\end{enumerate}

\section{Repetition Section of the Parameter Section}

\begin{figure}

\centerline{\epsfxsize=5.5in \epsffile{StarGui_repetitionSection.eps}}

\caption{
  The repetition section of the parameter section.
}
\label{repetitionSectionFig}
\end{figure}

Every repetition of a queryable parameter in the current stage is
represented by a section of the command dialog box.  See Figure
\ref{repetitionSectionFig}.
\begin{enumerate}
\item On the left is a widget which the user can use to select the
  value of the repetition.  This widget will be different for different
  types of parameters.  For instance, for a {\tt boolean} parameter,
  it may be a widget consisting of two radio buttons labelled ``Yes''
  and ``No'', whereas for a {\tt text} parameter it may be a
  text area.
\item On the right of the section is zero to three small buttons
  with icons in them.
\item The {\tt Move Up} button:
  \begin{enumerate}
  \item Is a small button with an up-arrow in it.
  \item Has the ToolTip ``Move up''.
  \item Appears only if
    {\tt MaxNumberOfReps} is not equal to 1 for this parameter, and
    {\tt RepsModel} for this parameter is {\tt sequence}.
  \item Is enabled only if this is not the first repetition (the topmost
    repetition).
  \end{enumerate}
\item The {\tt Move Down} button:
  \begin{enumerate}
  \item Is a small button with an down-arrow in it.
  \item Has the ToolTip ``Move down''.
  \item Appears only if
    {\tt MaxNumberOfReps} is not equal to 1 for this parameter, and
    {\tt RepsModel} for this parameter is {\tt sequence}.
  \item Is enabled only if this is not the last repetition (the bottommost
    repetition).
  \end{enumerate}
\item The {\tt Delete} button:
  \begin{enumerate}
  \item Is a small button with an X (``\verb/x/'') in it.
  \item Has the ToolTip ``Delete''.
  \item Appears only if {\tt MinNumberOfReps} is not equal to
    {\tt MaxNumberOfReps} for this parameter.
  \item Is enabled only if the current number of repetitions of the
    parameter is more than maximum(1, $m$), where $m$ is the value
    of {\tt MinNumberOfReps} for this parameter.
  \end{enumerate}
\end{enumerate}

\section{Question Dialog Box}

\begin{figure}

\centerline{\epsfxsize=3in \epsffile{StarGui_questionDialogBox.eps}}

\caption{
  The question dialog box.
}
\label{questionDialogBoxFig}
\end{figure}

See Figure \ref{questionDialogBoxFig}.  The Question Dialog Box
is used to ask {\tt Question}s.
\begin{enumerate}
\item On the top is the {\tt Label} attribute of the question.
\item Below that is a parameter occurrence selection widget, such as
  would appear in the Command Dialog Box for one repetition of a parameter.
\item On the lower left is a {\tt Cancel} button.
\item On the lower right is an {\tt OK} button.
\end{enumerate}

\section{Help Box}

\begin{figure}

\centerline{\epsfxsize=4in \epsffile{StarGui_helpBox01.eps}}

\caption{
  The help box, in the Top-Level state.
}
\label{helpBox01Fig}
\end{figure}

The Help Box has three states: the Top-Level state, the Command state, and
the Parameter/Question state.  It may be wise to put the contents of the window
into a JScrollPane, since the size of the contents will sometimes depend
on the size of the {\tt MultiLineHelp} attributes of given commands and
parameters.

\subsection{Top-Level State}

The Help Box begins in the Top-Level state.
See Figure \ref{helpBox01Fig}.
\begin{enumerate}
\item At the top is the word Commands.
\item Below that is a table (perhaps a JTable) with one row
  for every command in the application.  The table has two columns.
\item Each row of the table contains the {\tt Label} attribute of
  the command in the first column.  This should be displayed in a
  way that makes it clear that it is ``clickable''.
\item Each row of the table contains, in the second column, the
  {\tt OneLineHelp} attribute of the command.
\item At the bottom right is a button labelled OK.  If the user
  clicks this, the box should be disposed.
\end{enumerate}

\begin{figure}

\centerline{\epsfxsize=4in \epsffile{StarGui_helpBox02.eps}}

\caption{
  The help box, in the Command state.
}
\label{helpBox02Fig}
\end{figure}

\subsection{Command State}

If a user selects a command label when the help box is in the
Top-Level state, it moves to the Command state.
See Figure \ref{helpBox02Fig}.
\begin{enumerate}
\item At the top is the {\tt Label} attribute of the {\tt CommandGroup}
  that the command is in, an arrow, and the {\tt Label} attribute of the
  command.  (This gives an indication of where to find the command
  on the menus.)
\item Below that is a section containing the {\tt MultiLineHelp} attribute
  of the command.
\item Below that is a table (perhaps a JTable) with one row for each
  parameter in the command.  This parameter table has two columns.
\item Each row of the parameter table contains the {\tt Label} attribute of
  the parameter in the first column.  This should be displayed in a
  way that makes it clear that it is ``clickable''.
\item Each row of the parameter table contains, in the second column, the
  {\tt OneLineHelp} attribute of the parameter.
\item If there are any questions in the command, then below the parameter
  table, there is a JLabel reading ``Questions that might be asked:'',
  and a table with one row for each question in the command.
  This question table also has two columns.
\item Each row of the question table contains the {\tt Label} attribute of
  the question in the first column.  This should be displayed in a
  way that makes it clear that it is ``clickable''.
\item Each row of the question table contains, in the second column, the
  {\tt OneLineHelp} attribute of the question.
\item At the bottom left is a button labelled Back.  If the user
  clicks this, the box should go back to the Top-Level state.
\item At the bottom right is a button labelled OK.  If the user
  clicks this, the box should be disposed.
\end{enumerate}

\begin{figure}

\centerline{\epsfxsize=4in \epsffile{StarGui_helpBox03.eps}}

\caption{
  The help box, in the Parameter/Question state.
}
\label{helpBox03Fig}
\end{figure}

\subsection{Parameter/Question State}

If a user selects a parameter or question label when the help box is in the
Command state, it moves to the Parameter/Question state.
Figure \ref{helpBox03Fig} illustrates the Parameter/Question state when
a parameter has been selected; the state when a question has been selected
is similar.
\begin{enumerate}
\item At the top is the {\tt Label} attribute of the {\tt CommandGroup}
  that the command is in, an arrow, and the {\tt Label} attribute of the
  command.  (This provides continuity with the Command state.)
\item Below that is a section containing the text ``Parameter:'',
  followed by the {\tt Label} attribute of the parameter.
\item Below that is a section containing the {\tt MultiLineHelp} attribute
  of the parameter.
\item At the bottom left is a button labelled Back.  If the user
  clicks this, the box should go back to the Command state with the
  selected command.
\item At the bottom right is a button labelled OK.  If the user
  clicks this, the box should be disposed.
\item For a Question, the Help Box is identical, except that the text
  ``Parameter:'' reads ``Question:'' instead.
\end{enumerate}

\end{document}
