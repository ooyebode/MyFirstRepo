\documentclass[11pt]{article}

\setlength{\oddsidemargin}{0in}
\setlength{\evensidemargin}{0in}
\setlength{\topmargin}{-0.5in}
\setlength{\headheight}{0pt}
\setlength{\topskip}{0pt}
\setlength{\textwidth}{6.5in}
\setlength{\textheight}{9in}
%\setlength{\parindent}{0pt}
\setlength{\parskip}{1mm}

\usepackage{epsf}

% Magic incantation to allow proper PostScript to PDF conversion:
  \usepackage[T1]{fontenc} % type 1 fonts
  \usepackage{times}       % Times-Roman

\sloppy

\newcommand{\Fwname}{Johar}

\begin{document}

\begin{center} \bf \Large
Johar ``Star'' Interface Interpreter \\
Requirements Specification -- Behaviour
\end{center}

\section{Top-Level Behaviour}

\begin{enumerate}
\item Read the IDF.
\item Create the Main Panel.
  \begin{enumerate}
  \item Every {\tt CommandGroup} should correspond to a menu on the
    menu bar.
  \item There should also be the menu ``Star'', as the rightmost menu on the
    menu bar.
  \item The menu item for a command should be the {\tt Label}
    of the command, followed by ``{\tt ...}'' if selecting the 
    command will result in a Command Dialog being created (see below).
  \end{enumerate}
\item Initialize the Text Area to empty.
\item Create the GemSetting and validate it.
\item Call the application engine's initialization method.
\item Refresh the tables (see below).
\item Append a horizontal line (this could be just a line of minus
  characters) to the bottom of the Text Area.  (This will separate any
  initial message from the application engine from the messages that are
  responses to the commands.)
\item Call the {\tt ActiveIfMethod} method of every command, and make each menu
  item of each menu active or inactive (greyed out) according to the
  result of the {\tt ActiveIfMethod} method of the command.
\end{enumerate}

\section{Selecting a Command from a Menu}

When the user selects a command from a menu:
\begin{enumerate}
\item Set a String field {\tt lastDisplayedText} to the empty string.
  (Rationale:  this will be for displaying the last text that the program
  shows the user.  See below.)
\item If any table selection is {\it incomplete} (see below) for the command,
  then pop up a dialog box with an ``OK'' button telling the user
  which tables they have to select rows in in order to issue the
  command, and a description of the bounds of the number of rows to
  select (e.g., ``at least 2'', ``between 1 and 5'').
\item Otherwise, if no stage in the command has a queryable parameter
  (i.e., there are no parameters to any of the stages of the command, or all
  parameters in all stages of the command are
  {\tt tableEntry} parameters that are not queryable):
  \begin{enumerate}
  \item Call {\tt gemSetting.selectCurrentCommand(cmdName)}.
  \item Call {\tt gemSetting.selectCurrentStage(0)}.
  \item For each stage \verb/i/ in the command, perform the Stage Loop
    from the IntI Specification document.
  \item Perform the Question-and-Wrapup Procedure with parameters 0 and
    {\tt true}.  (This indicates that the Question-and-Wrapup
    Procedure should start with question number 0, and should wrap up
    the command if any questions are cancelled.)
  \end{enumerate}
\item Otherwise (i.e., if no table selection is incomplete for
  the command, but there are queryable parameters):
  \begin{enumerate}
  \item Call {\tt gemSetting.selectCurrentCommand(cmdName)}.
  \item Create the Command Dialog for the command.
    (The Command Dialog will take over the rest of the processing
    of the command.)
  \end{enumerate}
\end{enumerate}
A table selection is {\it incomplete for a given command}
if it contains at least one stage with a parameter with the following
characteristics:
\begin{enumerate}
\item The parameter is of type {\tt tableEntry};
\item The {\tt SourceTable} of the parameter is {\tt Browsable};
\item The parameter does not have a {\tt ParentParameter}; and
\item The {\tt MinNumberOfReps} for the parameter is greater than the
  number of rows that are currently selected in the {\tt SourceTable} of the
  parameter.
\end{enumerate}

\section{Question-and-Wrapup Procedure}

The Question-and-Wrapup Procedure takes two parameters: the number of the
question ({\tt int questionNumber}), and an indication of whether the
command should be wrapped up if a question is cancelled
({\tt boolean wrapUpIfCancelled}).

\begin{enumerate}
\item If {\tt questionNumber} is greater than or equal to the number
  of questions in the command, then this indicates that everything
  is OK for the command to be actually executed.
  \begin{enumerate}
  \item Call the command's {\tt CommandMethod}.
  \item Execute the Command Wrapup Procedure (see below) with parameter
    {\tt false} (indicating command was not cancelled).
  \item Return.
  \end{enumerate}
  Otherwise, continue with the below steps.
\item Call the {\tt AskIfMethod} for the question.
\item If the {\tt AskIfMethod} returns {\tt false}, then:
  \begin{enumerate}
  \item Execute the Question-and-Wrapup Procedure
    (recursively), using {\tt questionNumber + 1} and {\tt wrapUpIfCancelled}
    as parameters.
  \item Return.
  \end{enumerate}
  Otherwise, continue with the below steps.
\item If the question  has a {\tt DefaultValue}, set the value of
  (the response to) the question to the default value.
\item Otherwise, if the question has a {\tt DefaultValueMethod}, then call it
  and set the value of (the response to) the question to
  the default value.
\item Create a question dialog box for the question.
\end{enumerate}

\subsection{Question Dialog Cancel Button Action}

If the user presses the Cancel button on the Question Dialog Box:
\begin{itemize}
\item Dispose of the question dialog box.
\item If {\tt wrapUpIfCancelled}, then execute the Command Wrapup
  Procedure with parameter {\tt true}, indicating the command was
  cancelled.
\end{itemize}

\subsection{Question Dialog OK Button Action}

If the user presses the OK button on the Question Dialog Box:
\begin{enumerate}
\item Validate the current value of (the response to)
  the question as if it is a
  parameter, as indicated in requirements 26-51 of the IntI
  specification.  (A question with no default value, for which
  the user has not selected a value, should be interpreted as
  an invalid response.)
\item If the value of (the response to) the question does not pass
  validation:
  \begin{enumerate}
  \item Present an error message to the user in a dialog box.
    The error dialog box should have just an ``OK'' button.
  \item When the user presses OK, dispose of the error dialog box.
    (Rationale:  This will have the effect of returning control
    to the question dialog box, so that the user can select another
    value and press OK again or Cancel.)
  \end{enumerate}
\item Otherwise:
  \begin{enumerate}
  \item Load the current value of (the response to) the question
    into the Gem.
  \item Dispose of the question dialog box.
  \item Execute the Question-and-Wrapup Procedure
    (recursively), using {\tt questionNumber + 1} and {\tt wrapUpIfCancelled}
    as parameters.
  \end{enumerate}
\end{enumerate}

\section{Command Wrapup Procedure}

The Command Wrapup Procedure is called after the user selects a
command from the menu and some processing has been done.  It takes
one {\tt boolean} parameter,
called {\tt commandWasCancelled}.  {\tt commandWasCancelled}
should be {\tt false} only if the application engine method was called before
the command wrapup procedure was called.  If it is {\tt true}, this means that
the command was cancelled in some way by the user or by Star itself.
\begin{enumerate}
\item If there is a Command Dialog, dispose of it.
\item If {\tt commandWasCancelled} is {\tt false}:
  \begin{enumerate}
  \item Determine whether the application should quit, by checking
    the {\tt QuitAfter} attribute and/or calling the {\tt QuitAfterIfMethod}
    of the command.
  \item If the application should quit:
    \begin{enumerate}
    \item If {\tt lastDisplayedText} is non-null and not the empty string,
      then create a dialog box containing {\tt lastDisplayedText} and
      an ``OK'' button, and display it.  Wait for the user to
      click ``OK'', and then delete the dialog box.
      (Rationale:  if the command executed a {\tt showText} method and
      then Star exits, it may exit before the user has had time to
      read the text.)
    \item Exit the application, e.g. using {\tt System.exit(0)}.
    \end{enumerate}
  \item Otherwise:
    \begin{enumerate}
    \item Append a horizontal line to the Text Display Area, in order to
      separate the last command's output from the output of any
      future commands.
    \item Refresh the tables (see below).
    \end{enumerate}
  \end{enumerate}
\item Call the {\tt ActiveIfMethod} of every command, and make each
  menu item of each menu active or inactive (greyed out) according to
  the result of the {\tt ActiveIfMethod} of the command.
\end{enumerate}

\section{Refreshing the Tables}

To refresh the tables:
\begin{enumerate}
\item For each browsable table currently being displayed in the Table Area
  that is now hidden, delete the corresponding tab.
\item For each browsable table not currently being displayed in the Table
  Area that is now non-hidden, create a tab in the Table Area.
  (After this happens, it should be the case that the only tables
  with tabs in the Table Area are non-hidden tables.)
\item For each non-hidden table which has been updated since the
  last command execution terminated, update the data on the tab
  to reflect the contents of the table.
\item If there is a top table now, then place that table's tab on top
  in the tab pane.
\end{enumerate}

\section{The {\tt ShowTextHandler}}

The ShowTextHandler for Star handles text according to the prominence of
the text.
\begin{enumerate}
\item 0-1999:  For each line of text in the message:
  \begin{enumerate}
  \item Truncate the line of text, if necessary, to fit in the Status Bar.
    (Rationale:  it should fit all right, but just in case it doesn't, we
    can show part of it.  Because it is low-prominence, it seems OK to
    just show part of it.)
  \item Set the Status Bar to the text.
  \end{enumerate}
  [Note: This will cause the last line of the most recent message to
  overwrite anything that was in the Status Bar before.  This is OK
  because these messages are ``low prominence''.]
\item 2000-2999:  Append the text to the text being displayed in the
  Text Display Area.  Append the text also to {\tt lastDisplayedText}.
\item 3000 and higher:  Create a dialog box containing the text and an
  ``OK'' button, and display it.  When the user clicks the ``OK'' button,
  the box should be deleted.
\end{enumerate}

\section{The Command Dialog Box}

\subsection{Creating the Command Dialog Box}

\begin{enumerate}
\item Call {\tt gemSetting.selectCurrentStage(0)}.
\item Perform an Initialize Stage procedure.
\item While the current stage has no queryable parameters, perform
  a Next Stage procedure (see below).
\item Update the dialog box to reflect the current stage, as described
  in the Star GUI Specification document.
\end{enumerate}

\subsection{Initialize Stage Procedure}

\begin{enumerate}
\item If the current stage is a stage that has never been initialized so far,
  then for each parameter in the current stage:
  \begin{enumerate}
  \item If the parameter has a {\tt DefaultValue}, set the value of the
    parameter to the default value.
  \item Otherwise, if the parameter has a {\tt DefaultValueMethod}, then call it
    and set the value of the parameter to the default value.
  \end{enumerate}
\end{enumerate}

\subsection{Next Stage Procedure}

\begin{enumerate}
\item Perform a Wrap Up Stage procedure.
\item If the Wrap Up Stage procedure returns {\tt true}, then (assuming
  that the current stage is in the variable {\tt currentStage}):
  \begin{enumerate}
  \item Set the current stage to \verb/currentStage+1/.
  \item Call {\tt gemSetting.selectCurrentStage(currentStage)}.
  \item Perform an Initialize Stage procedure (see above).
  \end{enumerate}
\end{enumerate}

\subsection{Previous Stage Procedure}

\begin{enumerate}
\item Perform a Wrap Up Stage procedure.
\item If the Wrap Up Stage procedure returns {\tt true}, then (assuming
  that the current stage is in the variable {\tt currentStage}):
  \begin{enumerate}
  \item Set the current stage to \verb/currentStage-1/.
  \item Call {\tt gemSetting.selectCurrentStage(currentStage)}.
  \item Perform an Initialize Stage procedure (see above).
  \end{enumerate}
\end{enumerate}

\subsection{Wrap Up Stage Procedure}

This procedure takes no parameters.  It returns {\tt true} if all the tasks
that the user had to do in the current stage have been completed, and
it returns {\tt false} otherwise.
\begin{enumerate}
\item Validate the current values of the current repetitions of the
  parameters, as indicated in requirements 26-51 of the IntI
  specification.  (A parameter with no default value, for which the user
  has not selected a value, should be interpreted as not adding a valid
  repetition of the parameter.)
\item If any parameter or parameter repetition does not pass validation,
  then:
  \begin{enumerate}
  \item Collect information in a string about anything that does
    not pass validation.
  \item Present that information to the user in a dialog box.
    The dialog box should have just an ``OK'' button.
  \item When the user clicks OK:
    \begin{enumerate}
    \item If the current stage has no queryable
      parameters, then perform the Command Wrapup procedure, with the
      parameter {\tt true}.  (Rationale:  This might happen in some obscure
      situations, such as when a {\tt tableEntry} parameter has a parent
      parameter which gets set to the parent value.  In
      these situations, there is nothing we can do but cancel the command.
      The {\tt true} parameter to the Command Wrapup procedure indicates it has
      been cancelled.)
    \end{enumerate}
  \item Return {\tt false}.  (Rationale:  the
    user has not completed everything they have to do for this stage.)
  \end{enumerate}
\item (Otherwise:) Load the current values of the current repetitions of
  the parameters for the current stage into the Gem.
\item Call the ParameterCheckMethod of the current stage, if it has one.
\item If the ParameterCheckMethod exists and returns a non-null, non-empty
  string:
  \begin{enumerate}
  \item Show the user the string in a popup with an ``OK'' button.
  \item When the user clicks OK:
    \begin{enumerate}
    \item If the current stage has no queryable
      parameters, then perform the Command Wrapup procedure, with the
      parameter {\tt true}.
    \end{enumerate}
  \item Return {\tt false}.
  \end{enumerate}
\item Otherwise, return {\tt true}.
\end{enumerate}

\subsection{Next Button Action}

When the user presses the Next button (if it is not greyed out):
\begin{enumerate}
\item Perform a Next Stage procedure (see above).
\item While the current stage has no queryable parameters, perform
  a Next Stage procedure (see above).
\item Update the dialog box to reflect the current stage, as described
  in the Star GUI Specification document.
\end{enumerate}

\subsection{Previous Button Action}

When the user presses the Previous button (if it exists and is not greyed
out):
\begin{enumerate}
\item Perform a Previous Stage procedure (see above).
\item While the current stage has no queryable parameters, perform
  a Previous Stage procedure (see above).
\item Update the dialog box to reflect the current stage, as described
  in the Star GUI Specification document.
\end{enumerate}

\subsection{OK Button Action}

When the user presses the OK button (if it exists and is not greyed
out):
\begin{enumerate}
\item Perform the Wrap Up Stage procedure.
\item If the Wrap Up Stage procedure returns {\tt false}, then return.
\item For each stage \verb/i/ in the current command, from stage 0 to the
  last stage:
  \begin{enumerate}
  \item Call {\tt gemSetting.selectCurrentStage(i)}.
  \item Perform the Initialize Stage procedure.
  \item Perform the Wrap Up Stage procedure.
  \item If the Wrap Up Stage procedure returns {\tt false}, then return.
  \end{enumerate}
  (Rationale:  there may have been some previous stages which have
  become invalid as a result of the current stage; also, there may be
  stages after the current stage with parameters whose
  values have not been loaded yet.)
\item Perform the Question-and-Wrapup Procedure with parameters
  0 and {\tt false}.
\end{enumerate}

\subsection{Cancel Button Action}

When the user presses the Cancel button:
\begin{enumerate}
\item Perform the Command Wrapup procedure, with the parameter {\tt true}.
\end{enumerate}

\section{The Parameter Section}

\subsection{Add Another Button Action}

When the user clicks the Add Another button:
\begin{itemize}
\item Add another repetition section to the parameter section for the
  parameter, at the bottom of the list.  You may need to add a repetition
  to the model as well.
  (Rationale:  This is safe because if the Add Another button exists
  and is not greyed out, then we are not at the maximum number of
  repetitions for the parameter yet.)
\end{itemize}

\subsection{Move Up Button Action}

When the user clicks the Move Up button for repetition $k$:
\begin{itemize}
\item Exchange the value in repetition $k$ with the value in repetition $k-1$.
  This should be done in the model so that the changes will be automatically
  reflected in the GUI.
  (Rationale:  If the Move Up button exists and is not greyed out, then
  there is another repetition above the $k$th repetition.)
\end{itemize}

\subsection{Move Down Button Action}

When the user clicks the Move Down button for repetition $k$:
\begin{itemize}
\item Exchange the value in repetition $k$ with the value in repetition $k+1$.
  This should be done in the model so that the changes will be automatically
  reflected in the GUI.
  (Rationale:  If the Move Down button exists and is not greyed out, then
  there is another repetition below the $k$th repetition.)
\end{itemize}

\subsection{Delete Button Action}

When the user clicks the Delete button for repetition $k$:
\begin{itemize}
\item Delete repetition $k$.  This should be done in the model as well.
  (Rationale:  If the Delete button exists and is not greyed out, then
  we are not at the minimum number of repetitions for the parameter yet.)
\end{itemize}

\end{document}
